\documentclass{scrartcl}

\usepackage[utf8]{inputenc}
\usepackage[T1]{fontenc}
\usepackage{xspace}

\usepackage{hyperref}
\usepackage{enumitem}
\usepackage{colonequals}
\usepackage{stmaryrd}
\usepackage{tikz}
\usepackage{amsmath}

\usepackage{todonotes}

% style adjustments

\renewcommand{\arraystretch}{1.5}

% Manual Macros
\def\changemargin#1#2{}
\newcommand{\faq}[2]{\vspace{\topsep}\noindent\textbf{#1}\\\vspace{-\topsep}\begin{itemize}[nolistsep]\item[]{#2}\end{itemize}}


% Operations and Names

\newcommand{\Autosubst}{\textsc{Autosubst}\xspace}

\newcommand{\up}{\ensuremath{{}\mathop{\!\Uparrow\!}{}}}
\newcommand{\lift}{\text{\tiny\raisebox{0.15em}+\!\small{1}}}%\ensuremath{{}\mathop{\!\uparrow\!}{}}}
\newcommand{\scons}{\ensuremath{\coloncolon}}
\newcommand{\scomp}{\,\textrm{\guillemotright}\,}
\newcommand{\id}{\textsf{id}}
\newcommand{\subst}[1]{[#1]}

% remember TikZ positions.
\newcommand{\tmark}[1]{
  \tikz[remember picture, baseline, inner xsep=0, inner ysep=0.2em]{ \node [anchor=base] (#1) {\vphantom{M}};
}}%

% general stuff

\newcommand{\stackl}[2]{\vtop{\hbox{\strut#1}\hbox{\strut#2}}}
\newcommand{\stackc}[2]{\vtop{\setbox0\hbox{\strut #1}\copy0\hbox to\wd0{\hss\strut #2\hss}}}
\newcommand{\stackr}[2]{\vtop{\setbox0\hbox{\strut #1}\copy0\hbox to\wd0{\hss\strut #2}}}


% References

\bibliographystyle{plain}
\newcommand{\sref}[1]{{\tiny[#1]}}

% Lambda-terms

\newcommand{\ldot}{.\,}
\newcommand{\lam}[2][l0]{\lambda #2\!\tmark{#1}\,.\,}
\newcommand{\dlam}[1][l0]{\lambda\!\tmark{#1}\,\,}
\newcommand{\ind}[2][l0]{
  \tikz[remember picture, baseline=(i.base), inner xsep=0, inner ysep=0.2em]\node(i){$#2$};
  \tikz[remember picture, overlay]{
    \draw[->] (i.north) to[bend right] (#1.north);}}
\newcommand{\indb}[2][l0]{
  \tikz[remember picture, baseline=(i.base), inner xsep=0, inner ysep=0.2em]\node(i){$#2$};
  \tikz[remember picture, overlay]{
    \draw[->] (i.south) to[bend left] (#1.south);}}


% listings

\usepackage{listings}

\newcommand*{\lstitem}[1]{
  \setbox0\hbox{\lstinline{#1}}  
  \item[\usebox0]  
  % \item[\hbox{\lstinline{#1}}]
  \hfill \\
}

\newcommand{\lst}{\lstinline}
\lstdefinelanguage{Coq}%
  {morekeywords={Variable,Inductive,CoInductive,Fixpoint,CoFixpoint,%
      Definition,Lemma,Theorem,Axiom,Local,Save,Grammar,Syntax,Intro,%
      Trivial,Qed,Intros,Symmetry,Simpl,Rewrite,Apply,Elim,Assumption,%
      Left,Cut,Case,Auto,Unfold,Exact,Right,Hypothesis,Pattern,Destruct,%
      Constructor,Defined,Fix,Record,Proof,Induction,Hints,Exists,let,in,%
      Parameter,Split,Red,Reflexivity,Transitivity,if,then,else,Opaque,%
      Transparent,Inversion,Absurd,Generalize,Mutual,Cases,of,end,Analyze,%
      AutoRewrite,Functional,Scheme,params,Refine,using,Discriminate,Try,%
      Require,Load,Import,Scope,Set,Open,Section,End,match,with,Ltac,%
      Instance,%
      bind,as,%
      % , exists, forall
	},%
   sensitive, %
   morecomment=[n]{(*}{*)},%
   morestring=[d]",%
   literate={=>}{{$\Rightarrow$}}1 {>->}{{$\rightarrowtail$}}2{->}{{$\rightarrow$}}1
   {\/\\}{{$\wedge$}}1
   {|-}{{$\vdash$}}1
   {\\\/}{{$\vee$}}1
   {~}{{$\sim$}}1
   {exists}{{$\exists\!\!$}}1
   {forall}{{$\forall\!\!$}}1
   {sigma}{{$\sigma$}}1
   {theta}{{$\theta$}}1
   {tau}{{$\tau$}}1
   {rho}{{$\rho$}}1
   {xi}{{$\xi$}}1
   {exi}{exi}3 % insufficient patch to print reflexivity correctly
   {zeta}{{$\zeta$}}1
   {Gamma}{{$\Gamma$}}1
   {Delta}{{$\Delta$}}1
   {\\rhd}{{$\rhd$}}1
   %{>>}{\scomp}1
   %{<>}{{$\neq$}}1 indeed... no.
  }[keywords,comments,strings]%

\lstset{
   basicstyle=\ttfamily,
   keywordstyle=\bfseries\color{blue}
  }
\lstset{language=Coq}
\lstset{columns=fullflexible, keepspaces}

\begin{document}

\title{Autosubst Manual}
\date{\today}
\maketitle

\begin{abstract}
  Formalizing syntactic theories with variable binders is not easy. We present Autosubst, a library for the Coq proof assistant to automate this process. Given an inductive definition of syntactic objects in de Bruijn representation augmented with binding annotations, Autosubst synthesizes the parallel substitution operation and automatically proves the basic lemmas about substitutions. Our core contribution is an automation tactic that solves equations involving terms and substitutions. This makes the usage of substitution lemmas unnecessary. The tactic is based on our current work on a decision procedure for the equational theory of an extension of the sigma-calculus by Abadi et. al. The library is completely written in Coq and uses Ltac to synthesize the substitution operation.
\end{abstract}

\section{Tutorial}
\label{sec:Tutorial}

We start by importing the \Autosubst library.
\begin{lstlisting}
Require Import Autosubst.
\end{lstlisting}

Using de~Bruijn Syntax in Coq, the untyped lambda calculus is usually defined as shown Figure~\ref{fig:naive-ulc-term-def}.
\begin{figure}
\begin{minipage}{0.45\textwidth}
  \centering
\begin{lstlisting}
Inductive term : Type :=
  | TVar (x : nat)
  | App (s t : term)
  | Lam (s : term).
\end{lstlisting}
  \caption{Usual term definition with de~Bruijn indices}
  \label{fig:naive-ulc-term-def}
\end{minipage}
\hfill
\begin{minipage}{0.45\textwidth} 
  \begin{lstlisting}
Inductive term : Type :=
  | Var (x : var)
  | App (s t : term)
  | Lam (s : {bind term}). 
\end{lstlisting}
\caption{Term definition for \Autosubst}
\label{fig:autosubst-ulc-term-def}
\end{minipage}
\end{figure}
Using \Autosubst, we can automatically generate the substitution operation.
To do so, we annotate the positions of binders in the term type, since de Bruijn indices are interpreted differently if occuring below a binder. The annotated definition is shown in Figure~\ref{fig:autosubst-ulc-term-def}.
We write \lst${bind term}$ instead of \lst$term$ for the argument type of a constructor to indicate that this constructor serves as a binder for the argument. 
The type \lst${bind term}$ is definitionally equal to \lst$term$ and just serves as a tag interpreted while generating the substitution operation.
We also need to tag the constructor that builds variables. We do so by specifying the type of its single argument as \lst$var$, which is definitionally equal to \lst$nat$.

Using this definition of term, we can generate the substitution operation \lst@subst@ by declaring an instance of the \lst$Subst$ type class using our custom tactic \lst$derive$. This is comparable to the usage of deriving-clauses in Haskell.
We also need to define instances for the two auxiliary type classes \lst$Ids$ and \lst$Rename$, defining the generic functions \lst$ids$ and \lst$rename$. 
The function \lst$rename$ is only needed for technical reasons\footnote{The function rename applies a renaming \lst@var -> var@ to a term. While it is possible to give a direct structurally recursive definition of \lst@rename@, we need \lst@rename@ to give a structurally recursive definition of \lst@subst@. By unfolding \lst@subst@, it is possible to stumble upon an occurrence of \lst@rename@. We try to prevent this by eagerly replacing \lst@rename xi@ with \lst@subst (ren xi)@ in our custom tactics.} and is mostly hidden from the interface.
The function \lst$ids$ is the identity substitution, which is identical to the variable constructor.
\begin{lstlisting}
Instance Ids_type : Ids type. derive. Defined.
Instance Rename_type : Rename type. derive. Defined.
Instance Subst_type : Subst type. derive. Defined.
\end{lstlisting}

We can now use the pre-defined generic notations to call the just created substitution operation for \lst$term$. Given substitutions \lst$sigma$ and \lst$tau$, that is, values of type \lst$var -> term$, we can now write \lst$s.[sigma]$ for the application of \lst$sigma$ to  the term \lst$s$ and \lst$sigma >> tau$ for the composition of \lst$sigma$ and \lst$tau$. 
The notation \lst$s.[sigma]$ stands for \lst$subst sigma s$. 
The notation \lst$sigma >> tau$ stands for \lst$sigma >>> subst tau$, where \lst$>>>$ is function composition, i.e., \lst$(f >>> g) x = g(f(x))$.

Next, we generate the corresponding substitution lemmas by deriving an instance of the \lst$SubstLemmas$ type class.
\begin{figure}
\begin{lstlisting}
subst_comp s sigma tau : s.[sigma].[tau] = s.[sigma >> tau]
subst_id s : s.[ids] = s
id_subst x sigma : (ids x).[sigma] = sigma x
rename_subst xi s : rename xi s = s.[ren xi]
\end{lstlisting}
  \caption{Substitution Lemmas in \lst$SubstLemmas$}
  \label{fig:SubstLemmas}
\end{figure}
It contains the lemmas depiced in \autoref{fig:SubstLemmas}.
The lemma \lst$subst_comp$ states that instead of applying two substitutions in sequence, you can apply the composition of the two. This property is essential and surprisingly difficult to show if done manually. 
The lemma \lst$rename_subst$ is needed to eliminate occurrences of the renaming function \lst$rename$. Renaming can be expressed with ordinary substitutions using the function \lst$ren$ which lifts a function on variables \lst$var -> var$ to a substitution \lst$var -> term$. It satisfies \lst$ren xi = xi >>> ids$.
\begin{lstlisting}
Instance SubstLemmas_type : SubstLemmas type. gen_SubstLemmas. Qed.
\end{lstlisting}


This was all the boilerplate code needed to start using the library.
We can define the reduction relation for our untyped lambda terms as follows.
\begin{lstlisting}
Inductive step : term -> term -> Prop :=
| Step_Beta s s' t : s' = s.[t .: ids] -> step (App (Lam s) t) s'
| Step_App1 s s' t: step s s' -> step (App s t) (App s' t)
| Step_App2 s t t': step t t' -> step (App s t) (App s t')
| Step_Lam s s' : step s s' -> step (Lam s) (Lam s')
.
\end{lstlisting}
The most interesting rule is \lst$Step_Beta$, which expresses beta reduction using the stream-cons operator. Stream-cons \lst$(a .: f) : nat -> X$ for \lst$a : X$ and \lst$f : nat -> X$ satisfies the equations
\begin{align*}
  \text{\lst$(a .: f) 0$} &= \text{\lst$a$} \\
  \text{\lst$(a .: f) (S n)$} &= \text{\lst$f n$}
\end{align*}
So \lst$s.[t .: ids]$ is \lst$s$ where the index \lst$0$ is replaced by \lst$t$ and all other indices are reduced by one.
Also note that the rule \lst$Step_Beta$ contains a superfluous equation to make it applicable in more situations.


Now let us show a property of the reduction relation, the fact that it is closed under substitutions.
\begin{lstlisting}
Lemma step_subst s s' : step s s' -> forall sigma, step s.[sigma] s'.[sigma].
Proof. induction 1; constructor; subst; autosubst. Qed.
\end{lstlisting}
The automation tactic \lst$autosubst$ solves equations by normalizing them using a powerful rewriting system. 

We conclude this section with a detailed proof of type preservation for simply typed lambda calculus.
First, we need simple types. We define a base type \lst$Base$ and an arrow type \lst$Arr A B$ for functions from \lst$A$ to \lst$B$.
\begin{lstlisting}
Inductive type :=
| Base
| Arr (A B : type).
\end{lstlisting}
Then, we can define the typing judgment.
\begin{lstlisting}
Inductive ty (Gamma : var -> type) : term -> type -> Prop :=
| Ty_Var x A :     Gamma x = A -> 
                   ty Gamma (Var x) A
| Ty_Lam s A B :   ty (A .: Gamma) s B -> 
                   ty Gamma (Lam s) (Arr A B)
| Ty_App s t A B : ty Gamma s (Arr A B) -> ty Gamma t A -> 
                   ty Gamma (App s t) B.
\end{lstlisting}
We use infinite contexts. This allows us to encode contexts as functions of type \lst$var -> type$, which coincides with substitutions. Thus we can reuse the primitives and tactics of \Autosubst for the contexts.

Usually, a type preservation proof starts with a weakening lemma for the typing relation, which states that you can add a binding to the context. 
In de~Bruijn formalizations, it is usually stated with an operation that adds a single binding at an arbitrary position in the context.
Using parallel substitutions, we can generalize this to all contexts that can be obtained by reinterpreting the indices. This subsumes weakening, contraction and exchange.
\begin{lstlisting}
Lemma ty_ren Gamma s A: ty Gamma s A  -> forall Delta xi, 
                    Gamma = (xi >>> Delta)  ->
                    ty Delta s.[ren xi] A.
Proof.
  induction 1; intros; subst; asimpl; econstructor; eauto. 
  - eapply IHty. autosubst.
Qed.
\end{lstlisting}
The last application of \lst@autosubst@ solves a nontrivial equation between contexts built with \lst@.:@ and \lst@>>>@.

By generalizing \lst@ty_ren@ to substitutions, we obtain that we preserve typing if we replace variables by terms of the same type. Note that the proof uses \lst@ty_ren@, so it follows the structure of the definition of substitution.
\begin{lstlisting}
Lemma ty_subst Gamma s A: ty Gamma s A  -> forall sigma Delta,
                      (forall x, ty Delta (sigma x) (Gamma x))  ->
                      ty Delta s.[sigma] A.
Proof.
  induction 1; intros; subst; asimpl; eauto using ty. 
  - econstructor. eapply IHty.
    intros [|] *; asimpl; eauto using ty, ty_ren.
Qed.
\end{lstlisting}

To show type preservation of the simply typed lambda calculus, we use \lst@ty_subst@ to justify the typing of the result of the beta reduction.
The tactic \lst$ainv$ performs \lst$inversion$ on all hypothesis where this does not produce more than one subgoal.
\begin{lstlisting}
Lemma ty_pres Gamma s A : ty Gamma s A  -> forall s', 
                      step s s'  -> 
                      ty Gamma s' A.
Proof.
  induction 1; intros s' H_step; asimpl;
  inversion H_step; ainv; eauto using ty.
  - eapply ty_subst; try eassumption.
    intros [|]; simpl; eauto using ty.
Qed.      
\end{lstlisting}

\section{Reference Manual}
\label{sec:manual}

\subsection{Defining the Syntax}
\label{sec:defining-syntax}

To start using \Autosubst, you first have to define an inductive type of terms with de~Bruijn indices.
This should be a simple inductive definition without dependent types.
There must be at most one constructor for variables, aka de~Bruijn indices. It must have a single argument of type \lst$var$, which is a type synonym for \lst$nat$.
If a constructor acts as a binder for a variable of the term type \lst$T$ in a constructor argument of type \lst$U$, then \lst$U$ has to be replaced by \lst${bind T in U}$.
We can write \lst${bind T}$ instead of \lst${bind T in T}$.
\autoref{fig:term-type-example} shows how this looks for the two-sorted syntax of System F. 
\begin{figure}
  \centering
  \begin{lstlisting}
Inductive type : Type :=
| TyVar (x : var)
| Arr   (A B : type)
| All   (A : {bind type}).

Inductive term :=
| TeVar (x : var)
| Abs   (A : type) (s : {bind term})
| App   (s t : term)
| TAbs  (s : {bind type in term})
| TApp  (s : term) (A : type).
\end{lstlisting}
  \caption{Declaration of the syntax of System F}
  \label{fig:term-type-example}
\end{figure}

\subsection{Generating the Operations}
\label{sec:gener-oper}

We need to generate the substitution operations for the used term types and the corresponding lemmas.
This is done with instance declarations for the corresponding typeclass instances and the tactic \lst$derive$, which is defined as \lst$trivial with derive$ and we have collected a tactic for every typeclass in the hint database \lst$derive$. The operations are summarized in \autoref{tab:derived-ops} and the corresponding lemmas in \autoref{tab:derived-lemmas}.
\begin{table}
  \centering
  \begin{tabular}{l l l l}
  Typeclass                & Notation         & Definition           & Type                                   \\\hline\noalign{\vspace{0.5em}}
                             
  \lst$VarConstr term$     &                  & \lst$Var x$          & \lst$var -> term$                      \\
  \lst$Rename term$        &                  & \lst$rename xi s$    & \stackr{\lst$(var -> var) ->$}{\lst$term -> term$}     \\
  \lst$Subst term$         & \lst$s.[sigma]$  & \lst$subst sigma s$  & \stackr{\lst$(var -> term) ->$}{\lst$term -> term$}    \\
  \lst$HSubst term1 term2$ & \lst$s.|[sigma]$ & \lst$hsubst sigma s$ & \stackr{\lst$(var -> term1) ->$}{\lst$term2 -> term2$} 
\end{tabular}
  \caption{Operations that can be generated with \Autosubst}
  \label{tab:derived-ops}
\end{table}
\begin{table}
  \centering
  \begin{tabular}{l l}
    Typeclass & Contained Lemmas \\\hline\noalign{\vspace{0.5em}}
    
    \lst$SubstLemmas term$ & 
    \vtop{\hbox{\strut \lst$rename xi s = s.[ren xi]$,\quad \lst$s.[Var] = s$,}
          \hbox{\strut\lst$(Var x).[sigma] = sigma x$,\quad \lst$s.[sigma].[tau] = s.[sigma >>> tau]$}} \\
     \lst$HSubstLemmas term1 term2$ &
     \vtop{\hbox{\strut\lst$s.|[Var] = s$,\quad \lst$(Var x).|[sigma] = Var x$,}
           \hbox{\strut \lst$s.|[sigma].|[tau] = s.|[sigma >>> tau]$}} \\
    \lst$SubstHSubstComp term1 term2$ & \lst$s.[sigma].|[tau] = s.|[tau].[sigma >>| tau]$
  \end{tabular}
  \caption{Lemmas that can be generated with \Autosubst}
  \label{tab:derived-lemmas}
\end{table}

For example, the syntax of System F needs the declarations shown in \autoref{fig:derive-example}.
\begin{figure}
  \centering
\begin{lstlisting}
Instance VarConstr_type : VarConstr type. derive. Defined.
Instance Rename_type : Rename type. derive. Defined.
Instance Subst_type : Subst type. derive. Defined.

Instance SubstLemmas_type : SubstLemmas type. derive. Qed.

Instance HSubst_term : HSubst type term. derive. Defined.

Instance VarConstr_term : VarConstr term. derive. Defined.
Instance Rename_term : Rename term. derive. Defined.
Instance Subst_term : Subst term. derive. Defined.

Instance HSubstLemmas_term : HSubstLemmas type term. derive. Qed.
Instance SubstHSubstComp_type_term : SubstHSubstComp type term. derive. Qed.

Instance SubstLemmas_term : SubstLemmas term. derive. Qed.
\end{lstlisting}
  \caption{Declarations to derive the operations and lemmas for System F}
  \label{fig:derive-example}
\end{figure}
It is important to build the instances in the right order because they depend on each other.
We summarize the dependencies between the type class instances in \autoref{tab:decl-order}.
\begin{table}
  \centering
  \begin{tabular}{l l}
    Typeclass & Required Prior Declarations \\\hline\noalign{\vspace{0.5em}}

    \lst$VarConstr term$ & none \\

    \lst$Rename term$ & none \\

    \lst$Subst term$ & \vtop{
      \hbox{\strut
        \lst$Rename term$,
      }
      \hbox{\strut
        \lst$HSubst term' term$
      }
      \hbox{\strut
         \quad if \lst$term$ contains \lst${bind term' in term}$
      }
    } \\

    \lst$HSubst term1 term2$ & \vtop{
      \hbox{\strut
        \lst$Subst term1$,
      }
      \hbox{\strut
        \lst$HSubst term3 term4$
      }
      \hbox{\strut
         \quad if \lst$term2$ contains \lst${bind term3 in term4}$,
      }
      \hbox{\strut
        \lst$HSubst term1 term3$
      }
      \hbox{\strut
         \quad if \lst$term2$ contains \lst$term3$
      }
    } \\
    
    \lst$SubstLemmas term$ & \vtop{
      \hbox{\strut
        \lst$Ids term$,
      }
      \hbox{\strut
        \lst$Subst term$,
      }
      \hbox{\strut
        \lst$HSubstLemmas term1 term2$
      }
      \hbox{\strut
         and \lst$SubstHSubstComp term1 term2$
      }
      \hbox{\strut
       \quad if \lst$Subst term$ requires \lst$HSubst term1 term2$
    }} \\

    
    \lst$HSubstLemmas term1 term2$ &
    \vtop{
      \hbox{\strut
        \lst$HSubst term1 term2$,
      }\hbox{\strut
        \lst$SubstLemmas term1$
      }
    } 

  \end{tabular}
  \caption{Required Declaration Order of the Typeclass Instances}
  \label{tab:decl-order}
\end{table}

\subsection{Defined Operations}
\label{sec:pred-oper}

\Autosubst defines a number of operations, some of which depend on the generated operations.
They are important not only because they are useful in statements, but more importantly because our custom tactics incorporate facts about them. They are summarized in \autoref{tab:defined-ops}.

\begin{table}
  \centering
  \lstset{boxpos=t, aboveskip=0em, belowskip=0em}
  \begin{tabular}{l l r}
  Notation         & Definition                                       & Type                                   \\\hline\noalign{\vspace{0.5em}}
  
  \lst$f >>> g$    & \lst$fun x => g(f x)$                            & \stackr{\lst$forall A B C : Type, (A -> B) ->$}{\stackr{\lst$(B -> C) ->$}{\lst$A -> C$}} \\\noalign{\vspace{-1em}}
  \begin{lstlisting}
a .: f
  \end{lstlisting}
&\begin{lstlisting}
fun x => 
  match x with 
  | 0 => a 
  | S x' => f x' 
  end    
\end{lstlisting}
&
\begin{lstlisting}
forall X : Type,         X ->
           (var -> X) -> 
                var -> X
\end{lstlisting}  \\
  \lst$sigma >> tau$& \lst$sigma >>> subst tau$    & 
\stackr{\lst$(var -> term) ->$}{\stackr{\lst$(var -> term) ->$}{\lst$var -> term$}}     \\
\lst$sigma >>| theta$& \lst$sigma >>> hsubst theta$ & 
\stackr{\lst$(var -> term1) ->$}{\stackr{\lst$(var -> term2) ->$}{\lst$var -> term1$}} \\
\lst$ren xi$ & \lst$xi >>> ids$ & \stackr{\lst$(var -> var) ->$}{\lst$var -> term$} \\
\lst$(+ n)$ & \lst$fun x => n + x$ & \stackr{\lst$var ->$}{\lst$var -> var$}
\end{tabular}
  \caption{Defined Primitives of \Autosubst}
  \label{tab:defined-ops}
\end{table}

\subsection{The Automation Tactics}
\label{sec:lstautosubst-tactic}

Autosubst defines two automation tactics: \lst$asimpl$ and \lst$autosubst$.

\faq{\lst$asimpl$}{Normalizes the claim.}
\faq{\lst$asimpl in H$}{Normalizes the hypothesis \lst$H$.}
\faq{\lst$asimpl in *$}{Normalizes the claim and all hypothesis.}
\faq{\lst$autosubst$}{Normalizes the claim and tries to solve the resulting equation.}
\vspace{\topsep}

Both of them normalize the goal using a convergent rewriting system.
But while the interface and behavior of \lst$asimpl$ mimics \lst$simpl$, the closing tactic \lst$autosubst$ first normalizes an equational claim and then tries to prove it.
The rewriting system is an extension of the $\sigma$-calculus by Abadi et. al. \cite{abadi1991}.
Our goal is to solve all equations that hold without assumptions and are built using only our primitives, application and variables. At the moment, we hope to achieve this if \lst$(+ n)$ is only used with a constant \lst$n$. We consider ever real-world example of an unsolvable such equation a bug and invite you to submit it.



The normalization is done by interleaving the rewriting system with calls to \lst$simpl$ to incorporate the definitions of the derived operations. 


\section{Internals}
\label{sec:internals}

In the following, we describe technical challenges and how we solved them in Coq.

\subsection{Reducible Recursive Type Class Instances}
\label{sec:reduc-recurs-inst}

We need the substitution operation to reduce and simplify because there is no other way how our automation tactics could learn about the behavior of substitution on custom term types. However, this is challenging since the substitution operations we derive are instances of a type class. We needed a number of tricks to make this work smoothly.
\begin{itemize}
\item We use singleton type classes. So a type class instance is just a definition of the recursive procedure and the type class function reduces to its instance argument. This is important for two reasons. 
First, the guardedness checker does not unfold the record projections used for non-singleton type classes.
Second, when \lst$simpl$ reduces a definition bound to a \lst$fix$, it replaces all occurrences of this fix with the name of the definition afterwards. This also just works for singleton type classes.
\item All recursive calls are formulated using the function of the type class with the procedure name bound in the \lst$fix$-term serving as the implicit instance argument. This way, the result of the reduction of a type class function contains again calls to this type class function.
\item The Coq unification algorithm can perform unfoldings that are impossible with \lst$simpl$. This can lead to implicit instance arguments being unfolded. In turn, the type class inference can no longer infer instances depending on the unfolded instance. Apart from using \lst$simpl$ before using tactics that trigger unification like \lst$apply$ or \lst$constructor$, the only way to circumvent this is to revert the unfolding of instances. We automatically do this in all automation tactics by reinferring implicit instance arguments using \lst$exact _$.

\end{itemize}

\subsection{Generating the Operations Using Ltac}
\label{sec:gener-ltac}

We generate the renaming and substitution operations using Ltac. Since these are recursive functions, we use the tactics \lst$fix$ and \lst$destruct$. Consider the substitution operation for the term language from the tutorial. Our \lst$derive$ tactic constructs the following (proof) term.
\begin{lstlisting}
fix inst (sigma : var -> term) (s : term) {struct s} : term :=
  match s as t return (annot term t) with
  | Var x => sigma x
  | App s1 s2 => App s1.[sigma] s2.[sigma]
  | Lam s0 => Lam s0.[up sigma]
  end
\end{lstlisting}
Apart from the return annotation, which is an artifact of our approach, this looks quite clean. However, the recursive call is hidden in the implicit instance argument to \lst$subst$. We can see it if we show all implicit arguments.
\begin{lstlisting}
fix inst (sigma : var -> term) (s : term) {struct s} : term :=
  match s as t return (@annot Type term term t) with
  | Var x => sigma x
  | App s1 s2 => App (@subst term inst sigma s1) (@subst term inst sigma s2)
  | Lam s0 => Lam (@subst term inst (@up term Ids_term Rename_term sigma) s0)
  end
\end{lstlisting}

To construct this term with Ltac, we start with \lst$ fix inst 2$ to generate the \lst$fix$-term. Since we want to use the recursive identifier \lst$ident$ as a typeclass instance, we make it accessible to the instance inference by changing its type with 
\begin{lstlisting}
change _ with (Subst term) in inst  
\end{lstlisting}

Next, we need to construct the \lst$match$, which we can do with a \lst$destruct$. But then, the subgoals do not tell us the constructor of the current \lst$match$ case.
We get this information by annotating the goal with \lst$s$ before calling \lst$destruct$. Then this annotation contains the current constructor with all its arguments after the \lst$destruct$.
The function \lst$annot$, which is the identity on the first argument and ignores the second, allows us to perform the annotation. So we use the script \lst$intros xi s; change (annot term s); destruct s$
and then the claims of the subgoals are
\begin{itemize}
\item \lst$annot term (Var x)$
\item \lst$annot term (App s1 s2)$
\item \lst$annot term (Lam s0)$
\end{itemize}
So in effect, we have gained access to the patterns of the \lst$match$. Using a recursive, value-producing tactic, we can fold over the applied constructor like a list and change every argument depending on its type. The types of the arguments happen to contain the binding annotations in the definition of \lst$term$, so we can use an Ltac-\lst$match$ to read them and apply substitutions to the arguments accordingly. The type class inference automatically inserts the recursive call and the guardedness checker is able to unfold \lst$subst$ to see the applied recursive call.  

\section{Best Practices}
\label{sec:best-practices}



\subsection{Adding Primitives to \lst$autosubst$}
\label{sec:adding-prim}

If you want to add a new primitive and add support in \lst$autosubst$, you should do the following.

First, try to define the primitive using function composition or other existing primitives.
If this is possible, then you should define it using the notation mechanism to enable \lst$autosubst$ to simplify the function composition.
For example, if you want to lift a semantic interpretation to substitutions 
\begin{lstlisting}
subst_interp : (var -> value) -> (var -> term) -> var -> value  
\end{lstlisting}
then you should define
\begin{lstlisting}

Notation subst_interp rho sigma := sigma >> interp rho.
\end{lstlisting}
This automatically adds some limited support. To get full support, you can add the required lemmas to the autorewrite database \lst$autosubst$, which is used by the tactic \lst$autosubst$.


\section{Frequently Asked Questions}
\label{sec:faq}

\faq{I used \lst$simpl$ and now my goal contains strange identifiers.}
{The built-in tactic \lst$simpl$ can uncover implementation details. Use \lst$asimpl$ instead, which performs \lst$simpl$ and further simplifies the \Autosubst operations afterwards.}

\faq{I called \lst$constructor$ or \lst$apply$ and now the goal contains strange constructions that \lst$asimpl$ cannot hide.}
{These tactics sometimes perform aggressive reductions to match the goal that are not reachable with \lst$simpl$. You should try to call \lst$asimpl$ \emph{before} you call \lst$constructor$ or \lst$apply$ to make sure that no further reductions are needed.
}


\bibliography{bib}

\end{document}